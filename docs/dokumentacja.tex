\documentclass[12pt]{article}

\usepackage[polish]{babel}
\usepackage[T1]{fontenc}
\usepackage[utf8]{inputenc} % polish symbols

\usepackage{minted}

\newenvironment{code} {
\begin{minted}[mathescape,
               linenos,
               numbersep=5pt]{c}
    
\end{minted}
}

\title{Dokumentacja Projektu Kalkulator}
\author{Paweł Bielecki}
\date{\today}


\begin{document}
\maketitle

\section{Założenia projektu}

% \begin{minted}[mathescape,
%                linenos,
%                numbersep=5pt]{c}
% int main() {
%     printf("Hello World!");
% }
% \end{minted}

Celem projektu jest zaimplementowanie kalkulatora wykonującego działania na dowolnie dużych liczbach, w systemach liczbowych od 2 do 16.

Obsługiwane działania to:
\begin{itemize}
    \item dodawanie
    \item mnożenie
    \item potęgowanie
    \item dzielenie całkowitoliczbowe
    \item dzielenie modulo
    \item konwersja systemu liczbowego
\end{itemize}

\subsection{Używanie programu}
Program wczytuje działania z pliku tekstowego i wypisuje wyniki do innego pliku.

Uruchamiany poprzez:
\begin{minted}{bash}
    $ calc.exe <plik wejściowey> [plik wyjściowy]
\end{minted}

Bez podania pliku wyjściowego program tworzy plik wyjściowy z nazwy pliku wejściowego.

Dodatkowo program można uruchomić bez argumentów aby wejść w tryb interaktywny.
\begin{minted}{bash}
   $ calc.exe
   calc.exe repl
    Aby zakończyć: ctrl-d (ctrl-z na windowsie)

    [1]> + 10
    [2]> 15
    [3]> 20
    [4]> Wynik: 35
\end{minted}

W przypadku podania więcej niż 2 argumentów program wypisuje instrukcję używania.

\begin{minted}{bash}
   $ calc.exe a b c
   Użycie:
	calc.exe <ścieżka do pliku wejściowego>
    [ścieżka do pliku wyjściowego]

    Plik wejściowy musi być w formacie `*.txt`
    Plik wyjściowy jest opcjonalny
    (zostanie stworzony na podstawie nazwy pliku
    wejściowego `out_*.txt`) 
\end{minted}

\subsection{Obsługiwanie błędów}
Program wypisuje 3 rodzaje informacji dla użytkownika:

\begin{itemize}
    \item \textit{info} - informacja pomocnicza, program zadziałał poprawnie
    \item \textit{warning} - dane działanie wykonało się ale ze skutkiem możliwie niepożądanym
    \item \textit{error} - działanie nie zostało wykonane - w takim wypadku program wypisuje do pliku wyjściowego jedynie działanie i argumenty i przechodzi do kolejnego działania
\end{itemize}

Przykłady:
\begin{minted}{bash}
    $ calc.exe add.txt 
    INFO: Nie podano pliku wyjściowego, stworzono `out_add.txt`
\end{minted}

Jeżeli podamy za mało argumentów:
\begin{minted}{bash}
    WARNING [7:1]: Operacja przyjmuje przynajmniej 2 argumenty,
    otrzymano 1
	+ 10
	^
\end{minted}

W nawiasach kwadratowych mamy numer linii oraz number kolumny w której wystąpił błąd w pliku wejściowym.

\begin{minted}{bash}
    ERROR [1:4]: Baza może być tylko z przedziału [2, 16]
	+ 22
	  ^
\end{minted}

Obsługiwane błędy:
\begin{itemize}
    \item Nie można otworzyć pliku wejściowego
    \item Nie podano nazwy operacji (plik wejściowy zaczyna się od argumentu)
    \item Podano zbyt mało argumentów
    \item Nieoczekiwany znak
    \item Nieprawidłowy znak w liczbie o bazie x (np 123 nie jest poprawną liczbą w systemie dwójkowym)
    \item Przekroczona długość argumentu (40 znaków)
    \item Baza nie jest z przedziału [2, 16]
    \item Dzielenie przez 0
\end{itemize}

\section{Narzędzia}

Program został napisany w języku C, kompilowany w \textit{gcc}.
\begin{minted}{bash}
    $ gcc -pedantic -Wall -Wextra -g
\end{minted}

Używany edytor tekstu to \textit{neovim} (na linuxie) oraz \textit{visual studio code} (na windowsie).


\section{Ogólny opis rozwiązania}




\section{Szczegóły}
\section{Podsumowanie}

\end{document}
